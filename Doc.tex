\documentclass{memoir}



\usepackage{fontspec}
\setmainfont{Gurajada}


\usepackage{polyglossia}
\setmainlanguage{telugu}

\begin{document}

\title{శ్రీ గణేశ}

\maketitle

\chapter{గణేశ పఞ్చరత్న స్తోత్రము}


ముదాకరాత్త మోదకం సదావిముక్తి సాధకం \\
కలాధరావతంసకం విలాసిలోక రక్షకం \\
అనాయకైక నాయకం వనాశితేభ దైత్యకం\\
నతాశుభాశు నాశకం నమామి తం వినాయకం\\
\\
నతేతరాతి భీకరం నవోదితార్క భాస్వరం \\
నమత్సురారి నిర్జరం నతాధికాప దుర్ధరం\\
సురేశ్వరం నిధీశ్వరం గజేశ్వరం గణేశ్వరం \\
మహేశ్వరం తమాశ్రయే పరాత్పరం నిరంతరం\\
\\
సమస్త లోకశంకరం నిరన్త దైత్య కుంజరం\\
దరేతరోదరం వరం వరేభవక్త్రమక్షరం\\
కృపాకరం క్షమాకరం ముదాకరం యశస్కరం\\
మనస్కరం నమన్కృతాం నమస్కరోమి భాస్కరం\\
\\
అకించనాద్రిమార్జనం చిరన్తనోక్తి భాజనం \\
పురారి పూర్వనందనం సురారి గర్వచర్వణం \\
ప్రపంచ నాశభీషణం ధనంజయాది భూషణం \\
కపోలదాన వారణం భజే పురాణవారణం \\
\\
నితాన్తకాన్తదన్తకాన్తిమన్తకాన్తకాత్మజం \\
అచిన్త్యరూపమన్తహీనమన్తరాయకృన్తనం\\
హృదన్తరేనిరన్తరం వసంతమేవ యోగినాం\\
తమేక దన్తమేవతం విచిన్తయామి సన్తతం\\
\\
మహాగణేశ పఞ్చరత్నమాదరేణ యోన్వహమ్ \\
ప్రజల్పతి ప్రభాతకే హృదిస్మరన్మహేశ్వరమ్\\
అరోగతామదోషతాం సుసాహితీం సుపుత్రతాం \\
సమాహితాయురష్టభూతిమభ్యుపైతి సోచిరాత్\\


\chapter{శ్రీ లలితా సహస్ర నామ స్తోత్రము}

అస్య శ్రీ లలితాసహస్ర నామ స్తోత్రమాలా మంత్రస్య వసిన్యాది వాగ్దేవతా ఋషయః అనుష్ఠుప్ఛన్దః

\end{document}
